%%%%%%%%%%%%%%%%%%%%%%%%%%%%%%%%%%%%%%%%%
% University/School Laboratory Report
% LaTeX Template
% Version 3.1 (25/3/14)
%
% This template has been downloaded from:
% http://www.LaTeXTemplates.com
%
% Original author:
% Linux and Unix Users Group at Virginia Tech Wiki 
% (https://vtluug.org/wiki/Example_LaTeX_chem_lab_report)
% 
% License:
% CC BY-NC-SA 3.0 (http://creativecommons.org/licenses/by-nc-sa/3.0/)
%
%%%%%%%%%%%%%%%%%%%%%%%%%%%%%%%%%%%%%%%%%

%----------------------------------------------------------------------------------------
%	PACKAGES AND DOCUMENT CONFIGURATIONS
%----------------------------------------------------------------------------------------

\documentclass{article}

\usepackage[margin=0.5in]{geometry}
\usepackage{siunitx} % Provides the \SI{}{} and \si{} command for typesetting SI units
\usepackage{graphicx} % Required for the inclusion of images
\usepackage{natbib} % Required to change bibliography style to APA
\usepackage{amsmath} % Required for some math elements 

\setlength\parindent{0pt} % Removes all indentation from paragraphs

\renewcommand{\labelenumi}{\alph{enumi}.} % Make numbering in the enumerate environment by letter rather than number (e.g. section 6)

%\usepackage{times} % Uncomment to use the Times New Roman font

%----------------------------------------------------------------------------------------
%	DOCUMENT INFORMATION
%----------------------------------------------------------------------------------------

\title{Lossless Compression of Digital Audio Signals \\ ELE4ASP} % Title

\author{Joshua \textsc{Hickey}} % Author name

\date{\today} % Date for the report

\begin{document}

\maketitle % Insert the title, author and date

% If you wish to include an abstract, uncomment the lines below


\begin{abstract}

% Background
Data compression algorithms that do not approximate or truncate data during the reconstruction/deconstruction process are referred to as ``lossless''. Applications for lossless compression of audio data are largely for portability and convention, but is a practical solution for archiving and production. An example of a popular lossless audio codec is ``FLAC''.

% Results
Initially, a typical ``average linear predictor'' was implemented to generate non-optimal error data to be encoded and written to an output file. The encoding was performed by taking the error data (and after calculating the parameter $k$ based off the sum of squared prediction error) generating ``Rice codes'' for the error (based on this $k$ parameter).

% Conclusions
It was observed that more accurate prediction yields better compression performance, as there will be a lower sum of squared error in each block of data and therefore shorter ``Rice codes'' are able be generated. With poor prediction, the encoding algorithm generates unreliable codes due to overflows occuring while appending zeroes, which can be circumvented through the use of a higher $k$ parameter at the sacrifice of a severely limited compression performance.  

\end{abstract}

%----------------------------------------------------------------------------------------
%	SECTIONS
%----------------------------------------------------------------------------------------
\section*{Algorithm}

%% Include block diagram of process and explanation for each step



\cite{makhoul1975linear}
%----------------------------------------------------------------------------------------
%	BIBLIOGRAPHY
%----------------------------------------------------------------------------------------

\bibliographystyle{apalike}

\bibliography{refs}

%----------------------------------------------------------------------------------------


\end{document}
