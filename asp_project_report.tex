%%%%%%%%%%%%%%%%%%%%%%%%%%%%%%%%%%%%%%%%%
% University/School Laboratory Report
% LaTeX Template
% Version 3.1 (25/3/14)
%
% This template has been downloaded from:
% http://www.LaTeXTemplates.com
%
% Original author:
% Linux and Unix Users Group at Virginia Tech Wiki 
% (https://vtluug.org/wiki/Example_LaTeX_chem_lab_report)
% 
% License:
% CC BY-NC-SA 3.0 (http://creativecommons.org/licenses/by-nc-sa/3.0/)
%
%%%%%%%%%%%%%%%%%%%%%%%%%%%%%%%%%%%%%%%%%

%----------------------------------------------------------------------------------------
%	PACKAGES AND DOCUMENT CONFIGURATIONS
%----------------------------------------------------------------------------------------

\documentclass{article}

\usepackage{geometry}
\usepackage{listings}
\usepackage{siunitx} % Provides the \SI{}{} and \si{} command for typesetting SI units
\usepackage{graphicx} % Required for the inclusion of images
\usepackage{natbib} % Required to change bibliography style to APA
\usepackage{amsmath} % Required for some math elements 

\setlength\parindent{0pt} % Removes all indentation from paragraphs

\renewcommand{\labelenumi}{\alph{enumi}.} % Make numbering in the enumerate environment by letter rather than number (e.g. section 6)

%\usepackage{times} % Uncomment to use the Times New Roman font

%----------------------------------------------------------------------------------------
%	DOCUMENT INFORMATION
%----------------------------------------------------------------------------------------

\title{Lossless Compression of Digital Audio Signals \\ ELE4ASP} % Title

\author{Joshua \textsc{Hickey}} % Author name

\date{\today} % Date for the report

\begin{document}

\maketitle % Insert the title, author and date

% If you wish to include an abstract, uncomment the lines below


\begin{abstract}

% Background
Data compression algorithms that do not approximate or truncate data during the reconstruction/deconstruction process are referred to as ``lossless''. Applications for lossless compression of audio data are largely for portability and convention, but is a practical solution for archiving and production. An example of a popular lossless audio codec is ``FLAC''.\\

% Results
Fixed coefficient linear predictors (non-optimal) were implemented to generate error data to be encoded, compressed and written to an output file. The encoding was performed by taking the error signal (and after calculating the parameter $k$ based off the sum of squared prediction error) generating ``Rice codes'' for the error (based on this $k$ parameter). Basic dat compression was performed by writing only descriptive bits in the encoded symbol to the output file.\\

% Conclusions
It was observed that more accurate prediction yields better compression performance, as there will be a lower sum of squared error in each block of data and therefore shorter ``Rice codes'' are generated. With poor prediction, the encoding algorithm generates inefficient symbols due to a high amount of appending zeroes (caused by prediction error signals with high value ``upper-nibbles''. Data expansion due to this problem can be circumvented through the use of a higher $k$ parameter at the sacrifice of compression performance in shorter encoded symbols.\\

\end{abstract}

%----------------------------------------------------------------------------------------
%	SECTIONS
%----------------------------------------------------------------------------------------
\section*{Algorithm}

\subsection*{Prediction}
\cite{makhoul1975linear}
% Average predictor
\begin{equation}
S_p(i) = \sum_{j=1}^{N=2} 0.5s(i-j) \\
\end{equation}

\begin{equation}
S_p(i) = s(i-1) \\
\end{equation}

\begin{equation}
S_p(i) = 2s(i-1) - s(i-2) \\
\end{equation}

\begin{equation}
S_p(i) = 3s(i-1) - 3s(i-2) + s(i-3) \\
\end{equation}

\subsection*{Rice Coding}

\begin{center}
\begin{tabular}{|r|c|c|l|}
  \hline
  \textbf{1} & $X$ (no. of zeroes) & \textbf{1} & last $k$ bits \\
  \hline
\end{tabular}
\end{center}

\section*{Implementation}
%Block diagram of implementation of whole process
%Read block
%predict block
%encode prediction error vector
%compress encoded data
%write compressed data
%repeat process for incomplete block

\section*{Results}

The following experiment was conducted using a 2562 byte .wav file, using a range of block sizes in an attempt to observe correlating features between compression efficiency and parameters calculated using block size or attributes of the data contained within the block.

\begin{center}
\begin{table}  
  \begin{tabular}{|c|c|c|}
    \hline
    \textbf{Input file size} & \textbf{Output file size} & \textbf{Block size} & \textbf{Average K} \\
    \hline
    2562 & 2282 & 1700 \\
    2562 & 2425 & 1500 \\
    2562 & 2779 & 1400 \\
    2562 & 2395 & 900 \\
    2562 & 2858 & 200 \\
  \end{tabular}
  \caption{Table of results comparing file sizes in bytes}
  \label{tab:results}
\end{table}

Compression performance varied $k$

\section*{Conclusion}
Given an accurate prediction model of the input signal, ``Rice codes'' are a viable and efficient solution to losslessly compress data by enabling the description of a signal sample in shorter bit widths than the literal value. \\ 

The prediction model is pivotal in increasing the compression performance of the encoder by minimizing prediction errors and therefore allowing the use of smaller symbols to describe the intact input signal. \\

Reliability of the symbols was also noted to be important, as the signal could not be reconstructed in the event of overflows, typically observed when a large sum of squared errors designates a non ideal $k$ parameter for encoding the prediction error. Some data expansion was observed in this case, as the non ideal $k$ parameter sometimes generated codes larger than the original sample. \\ 

\newpage
\appendix
\newgeometry{left=2cm}
\section*{Source Code} \label{App:code}
% the \\ insures the section title is centered below the phrase: AppendixA
\subsection*{Main Routine}
\lstinputlisting{asp_project.c}
\newpage
\subsection*{Prediction}
\lstinputlisting{prediction.c}
\lstinputlisting{prediction.h}
\newpage
\subsection*{Rice Coding}
\lstinputlisting{encoder.c}
\lstinputlisting{encoder.h}
\newpage
\subsection*{File IO}
\lstinputlisting{file_io.c}
\lstinputlisting{file_io.h}
%----------------------------------------------------------------------------------------
%	BIBLIOGRAPHY
%----------------------------------------------------------------------------------------

\bibliographystyle{apalike}

\bibliography{refs}

%----------------------------------------------------------------------------------------


\end{document}
